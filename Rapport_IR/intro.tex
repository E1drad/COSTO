\topskip0pt
\vspace*{\fill}
\chapter*{Introduction}
\label{chap:introduction}
Ce rapport reprend le travail qui a été effectué dans le cadre du cours d'Introduction à la Recherche du master 1 ALMA 2017 par notre groupe composé de Montalvo Araya, Charles-Eric Begaudeau et Charlène Servantie.


Le sujet sur lequel nous avons travaillé est : "Faut-il tester le modèle ou le code des composants ?", un sujet proposé par Pascal André, Jean-Marie Mottu et Gilles Ardourel de l'équipe AeLoS\footnote{Architectures et Logiciels Sûrs, \url{https://ls2n.fr/equipe/AeLoS/}}, qui ont été nos encadrants au cours de cette découverte du monde de la recherche.

Nous avons tout d'abord choisi de présenter le modèle Kmelia, le framework COSTO et les concepts abordés dans le chapitre~\ref{chap:KmeliaCOSTO}, à travers notamment les différents articles que nous avons été amenés à lire pour appréhender le sujet de recherche. Ensuite nous allons parler dans le chapitre~\ref{chap:harnais} sur le test de modèle à travers le générateur de harnais de COSTOTest, puis nous présenterons l'utilisation de JUnit pour faire du test sur le code Java généré et notre processus de travail pour générer ces tests dans le chapitre~\ref{chap:JUnit}. Nous avons ensuite comparé le test de modèle avec le test de code dans le chapitre~\ref{chap:comparaison}, où nous discutons également de notre travail et de nos blocages au cours de ce travail. Enfin, nous avons essayé d'apporter des réflexions sur de nouveaux tests et l'analyse de mutation au chapitre~\ref{chap:mutations}

%alias la partie 4, bref, lien vers la partie
  %plan
  %modèle kmelia
  %générateur de harnais de test pour le model testing
  %test java JUnit focus sur processus + méthodologie
  %comparaison model testing/ testing junit
  %mutations / autre exemple, autre service
 

\vspace*{\fill}